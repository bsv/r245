Программа audit предоставляет графический интерфейс пользователя для взаимодействия с считывателем 
PR-G07 компании Parsec. Считыватель регистрирует происходящие события в специальной памяти,
называемой журналом транзакций. Программа считывает информацию из этого журнала и предоставляет её 
пользователю в виде таблицы в окне монитора, по которой можно осуществлять поиск необходимых данных. Результаты поиска,
при желании, возможно сохранить в файл отчета в формате csv, который без проблем открывается в Microsoft 
Office Excel.

Audit позволяет производить настройку устройства считывателя в окне настроек: задавать время реакции и дальность считывания
для 2 каналов, проводить синхронизацию внутренних часов, включать или отключать имеющиеся каналы.
Для настройки доступны устройства, параллельно подключенные к компьютеру через USB интерфейс. 

Для удобства обработки полученной от считывателя информации, можно настроить реакцию на определенные 
события. Например, задать цвет, которым будет выделяться соответствующая запись в таблице монитора, или 
настроить оповещение оператора специальным сообщением, которое будет показываться при наступлении
определенных обстоятельств (обнаружение метки и т.п.). 

Все настройки, сделанные в программе, сохраняются в отдельном файле. Также сохраняется вся история 
событий, полученных от считывателя, и данные о реакциях оператора на сообщения. При закрытии программа
запоминает открытые файлы, а при открытии открывает их, тем самым восстанавливая рабочий сеанс пользователя.